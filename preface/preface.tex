Photonic devices are important to a wide range of applications, from communications to fundamental science.  However, the design of such devices is traditionally done by hand, using physical intuition and trial and error.  The development of \textit{inverse design}, where computational optimization techniques are used to design devices based on certain specifications, has led to the discovery of many compact, non-intuitive structures with superior performance. Among various methods, large-scale, gradient-based optimization techniques have been one of the most important ways to design a structure containing a vast number of degrees of freedom. These techniques are made possible by the \textit{adjoint method}, in which the gradient of an objective function with respect to all design degrees of freedom can be computed using only two full-field simulations.

In this thesis, we will discuss the application of inverse design to two emerging photonic technologies and discuss the generalization of the adjoint method to new scenarios.   First, we will present the inverse design of laser-driven particle accelerators on a chip as well as our efforts to scale this technology to higher energy gains using photonic integrated circuits.  Next, we will discuss how the adjoint method may be used to perform on-chip training of optical neural networks.  We will show that this technique corresponds to a physical implementation of the backpropagation algorithm, commonly used in traditional neural networks.  A procedure for measuring the gradient determined by the adjoint method will be introduced.  Then, we will discuss the application of this technique to design general wave systems capable of perform machine learning computation on sequence data in the form of time series signals.  Finally, we will discuss the generalization of the adjoint method to nonlinear optical phenomena and show that this may be used to devise compact photonic switches in a Kerr nonlinear material.
